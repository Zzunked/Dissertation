
{\actuality} 
\paragraph{Актуальность темы исследования.}
В современной физике поверхности одной из основных задач является
создание низкоразмерных систем. После того, как графен был предсказан
теоретически~\cite{Wallace1947,McClure1956,Slonczewski1958}, а позже и получен эксперементально~\cite{Novoselov2004}, интерес к 
низкоразмерным системам резко возрос среди ученых. Причиной этому является
серьезный потенциал для использовании в перспективных низкоразмерных 
устройствах, таких как квантовые компьютеры~\cite{Barends2014}, молекулярные переключатели~\cite{G.Joachim2000,Nitzan2003}. 
К наиболее интересным низкоразмерным нанообъектам относятся структуры на 
основе двумерных гексагональных кристаллов графена и нитрида бора.
Данная работа посвящена гексагональному нитриду бора(h-BN). 


Как и графен, гексагональный нитрид бора может быть получен путем каталитического разложения молекул прекурсора из газообразной 
фазы на активной подложке(метод CVD), в случае h-BN это
боразин~\cite{Kidambi2014,Paffett1990}. Нитрид бора выращенный на
металлической подложке методом CVD может периодически деформироваться, это показано в работах~\cite{Preobrajenski2008_Adsorption-inducedgapstatesofh-BNonmetalsurfaces,Brugger2009,Nagashima1995,Weng2016}


{\aim} данной работы является исследование электронной и магнитной структур легированных металлами топологических изоляторов с помощью \textbf{методов} фотоэлектронной спектроскопии с угловым разрешением и SQUID магнитометрии . 

% Для~достижения поставленной цели необходимо было решить следующие {\tasks}:
% \begin{enumerate}
%   \item Исследовать влияние ультратонких пленок Pb на электронную структуру топологических изоляторов $ Bi_{2}Se_{3} $ и $ Sb_{2}Te_{3} $ и сравнить полученные результаты с имеющимися литературными данными о свойствах топологических изоляторов, легированных немагнитными металлическими примесями.
%   \item Исследовать электронную структуру допированных редкоземельными металлами (Gd, Dy, Er) топологических изоляторов со стехиометрией $ Bi_{x}Sb_{2-x}Te_{3} $ и проанализировать возможности реализации щели в топологических состояниях Дираковского конуса, обусловленной магнитными свойствами легирующей примеси.
%   \item Исследовать электронную структуру магнито-допированного топологического изолятора со стехиометрией $ Bi_{1.09}Gd_{0.06}Sb_{0.85}Te_{3} $ и путем сравнения с магнитной структурой выявить связи между электронными и магнитными свойствами в данной системе.
% \end{enumerate}


%{\novelty}
%\begin{enumerate}
%  \item Впервые \ldots
%  \item Впервые \ldots
%  \item Было выполнено оригинальное исследование \ldots
%\end{enumerate}

%{\influence} \ldots

%{\methods} \ldots

%{\defpositions}
%\begin{enumerate}
%  \item Первое положение
%  \item Второе положение
%  \item Третье положение
%  \item Четвертое положение
%\end{enumerate}
%В папке Documents можно ознакомиться в решением совета из Томского ГУ
%в~файле \verb+Def_positions.pdf+, где обоснованно даются рекомендации
%по~формулировкам защищаемых положений. 

%{\reliability} полученных результатов обеспечивается \ldots \ Результаты находятся в соответствии с результатами, полученными другими авторами.


%{\probation}
%Основные результаты работы докладывались~на:
%перечисление основных конференций, симпозиумов и~т.\:п.

%{\contribution} Автор принимал активное участие \ldots

%\publications\ Основные результаты по теме диссертации изложены в ХХ печатных изданиях~\cite{Sokolov,Gaidaenko,Lermontov,Management},
%Х из которых изданы в журналах, рекомендованных ВАК~\cite{Sokolov,Gaidaenko}, 
%ХХ --- в тезисах докладов~\cite{Lermontov,Management}.

%\ifnumequal{\value{bibliosel}}{0}{% Встроенная реализация с загрузкой файла через движок bibtex8
%    \publications\ Основные результаты по теме диссертации изложены в XX печатных изданиях, 
%    X из которых изданы в журналах, рекомендованных ВАК, 
%    X "--- в тезисах докладов.%
%}{% Реализация пакетом biblatex через движок biber
%Сделана отдельная секция, чтобы не отображались в списке цитированных материалов
%    \begin{refsection}[vak,papers,conf]% Подсчет и нумерация авторских работ. Засчитываются только те, которые были прописаны внутри \nocite{}.
        %Чтобы сменить порядок разделов в сгрупированном списке литературы необходимо перетасовать следующие три строчки, а также команды в разделе \newcommand*{\insertbiblioauthorgrouped} в файле biblio/biblatex.tex
 %       \printbibliography[heading=countauthorvak, env=countauthorvak, keyword=biblioauthorvak, section=1]%
 %       \printbibliography[heading=countauthorconf, env=countauthorconf, keyword=biblioauthorconf, section=1]%
 %       \printbibliography[heading=countauthornotvak, env=countauthornotvak, keyword=biblioauthornotvak, section=1]%
 %       \printbibliography[heading=countauthor, env=countauthor, keyword=biblioauthor, section=1]%
%        \nocite{%Порядок перечисления в этом блоке определяет порядок вывода в списке публикаций автора
 %               vakbib1,vakbib2,%
 %               confbib1,confbib2,%
 %b              bib1,bib2,%
 %       }%
 %       \publications\ Основные результаты по теме диссертации изложены в~\arabic{citeauthor}~печатных изданиях, 
 %       \arabic{citeauthorvak} из которых изданы в журналах, рекомендованных ВАК, 
 %       \arabic{citeauthorconf} "--- в~тезисах докладов.
 %   \end{refsection}
%    \begin{refsection}[vak,papers,conf]%Блок, позволяющий отобрать из всех работ автора наиболее значимые, и только их вывести в автореферате, но считать в блоке выше общее число работ
%        \printbibliography[heading=countauthorvak, env=countauthorvak, keyword=biblioauthorvak, section=2]%
%        \printbibliography[heading=countauthornotvak, env=countauthornotvak, keyword=biblioauthornotvak, section=2]%
%        \printbibliography[heading=countauthorconf, env=countauthorconf, keyword=biblioauthorconf, section=2]%
%        \printbibliography[heading=countauthor, env=countauthor, keyword=biblioauthor, section=2]%
%        \nocite{vakbib2}%vak
%        \nocite{bib1}%notvak
%        \nocite{confbib1}%conf
%    \end{refsection}
%}
%При использовании пакета \verb!biblatex! для автоматического подсчёта
%количества публикаций автора по теме диссертации, необходимо
%их~здесь перечислить с использованием команды \verb!\nocite!.
