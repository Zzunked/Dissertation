
{\actuality} 
В современной физике поверхности одной из основных задач является
создание низкоразмерных систем. После того, как графен был предсказан
теоретически~\cite{Wallace1947,McClure1956,Slonczewski1958}, а позже получен экспериментально~\cite{Novoselov2004}, интерес к 
низкоразмерным системам резко возрос среди ученых. Причиной этому является
серьезный потенциал для использовании в перспективных низкоразмерных 
устройствах, таких как квантовые компьютеры~\cite{Barends2014}, молекулярные переключатели~\cite{G.Joachim2000,Nitzan2003}. 
К наиболее интересным низкоразмерным нанообъектам относятся структуры на 
основе двумерных гексагональных кристаллов графена и нитрида бора.
Гексагональный нитрид бора (h-BN) имеет огромный потенциал применения в различных
электрооптических приборах, например, ультрафиолетовые лазеры~\cite{Kubota2007}, 
свето-диоды, фотодетекторы~\cite{Zheng2008} и другие электронные
устройства, в основе которых лежат квантоворазмерные эффекты~\cite{
Chau2007}. Также в последнее время h-BN все больше привлекает внимание
ученых как изолятор в конфигурациях вида graphene/h-BN/metal~\cite{
Kamalakar2014,Kamalakar2016,Usachev_doc}.
Данная работа посвящена гексагональному нитриду бора. 


Как и графен, гексагональный нитрид бора может быть получен путем каталитического разложения молекул прекурсора из газообразной 
фазы на активной подложке (метод CVD). В случае h-BN используется
боразин~\cite{Kidambi2014,Paffett1990}. Нитрид бора, выращенный на
монокристаллической подложке методом CVD, может деформироваться
переодическим образом, это показано для h-BN на Pt(111)~\cite{Cavar2008},
Ir(111)~\cite{FarwickzumHagen2016,Usachov2012, Orlando2012},
Ru(0001)~\cite{Brugger2009}, Fe(110)~\cite{Vinogradov2012} и Pd(111)~\cite{Nagashima1995}. Во всех этих случаях возникает структура муара, а в системах
с сильным взаимодействием между h-BN и подложкой образовывается сильно 
корругированая наносетка~\cite{Preobrajenski2008_Adsorption-inducedgapstatesofh-BNonmetalsurfaces}. 
Также интересно, как адсорбированные и интеркалированные частицы, например,
атомы кислорода, влияют на кристаллическую и электронную структуру 
интерфейса h-BN/металл. Исследования в этом направлении проведены для 
интерфейсов h-BN/Ni(111)~\cite{Wasey2013}, h-BN/Pt(111)~\cite{Meng2018},
h-BN/Ir(111)~\cite{Simonov2012_h-BN/Ir_Oxydation}.


Таким образом, гексагональный нитрид бора, выращенный на разных 
подложках, хорошо изучен, однако интерфейс h-BN/Co остается малоизученным.
Кобальт очень интересен тем, что постоянные решеки h-BN и Co(0001) практически
совпадают~\cite{Usachov2018_h-BN/_PED_STM}, а это означает, что можно получить
интерфейс со структурой $(1\times1)$ высокого качества, как, например, graphen/Co(0001)~\cite{Usachov2016}. Также кобальт является химически активным металлом
и очень хорошо вступает в реакцию с кислородом. Поэтому в настоящей работе 
изучался интерфейс именно h-BN/Co(0001) при взаимодействии с кислородом.


{\aim} данной работы является исследование влияния молекулярного кислорода на 
интерфейс h-BN/Co(0001). В работе использовались поверхностно-чувствительные методы рентгеновской фотоэлектронной спектроскопии (
РФЭС), дифракции медленных электронов (ДМЭ) и рентгеновской 
спектроскопии поглощения в области ближней тонкой структуры (Near-Edge 
X-Ray Absorption Fine Structure Spectroscopy, NEXAFS
spectroscopy). 


Для достижения поставленной цели необходимо было решить следующие {\tasks}:
\vspace{15pt}
\begin{enumerate}
  \item Приготовление высококачественного монослоя h-BN на
  монокристаллической поверхности Co(0001) и его характеризация 
  методами спектроскопии и дифракции медленных электронов.
  \item Осуществление поэтапного прогрева системы h-BN/Co(0001) в атмосфере молекулярного кислорода.
  \item Исследование состояния системы на каждом этапе прогерва
  методами спектроскопии.
  \item Анализ полученных спектров поглощения и фотоэмиссии 
  остовных уровней и получение информации о механизме 
  взаимодействия молекулярного кислорода и системы h-BN/Co(0001).
  \item Теоретический расчет структуры гексагонального  нитрида бора со встроенными атомами кислорода.
\end{enumerate}
\vspace{15pt}

%{\novelty}
%\begin{enumerate}
%  \item Впервые \ldots
%  \item Впервые \ldots
%  \item Было выполнено оригинальное исследование \ldots
%\end{enumerate}

%{\influence} \ldots

%{\methods} \ldots

%{\defpositions}
%\begin{enumerate}
%  \item Первое положение
%  \item Второе положение
%  \item Третье положение
%  \item Четвертое положение
%\end{enumerate}
%В папке Documents можно ознакомиться в решением совета из Томского ГУ
%в~файле \verb+Def_positions.pdf+, где обоснованно даются рекомендации
%по~формулировкам защищаемых положений. 

%{\reliability} полученных результатов обеспечивается \ldots \ Результаты находятся в соответствии с результатами, полученными другими авторами.


%{\probation}
%Основные результаты работы докладывались~на:
%перечисление основных конференций, симпозиумов и~т.\:п.

%{\contribution} Автор принимал активное участие \ldots

%\publications\ Основные результаты по теме диссертации изложены в ХХ печатных изданиях~\cite{Sokolov,Gaidaenko,Lermontov,Management},
%Х из которых изданы в журналах, рекомендованных ВАК~\cite{Sokolov,Gaidaenko}, 
%ХХ --- в тезисах докладов~\cite{Lermontov,Management}.

%\ifnumequal{\value{bibliosel}}{0}{% Встроенная реализация с загрузкой файла через движок bibtex8
%    \publications\ Основные результаты по теме диссертации изложены в XX печатных изданиях, 
%    X из которых изданы в журналах, рекомендованных ВАК, 
%    X "--- в тезисах докладов.%
%}{% Реализация пакетом biblatex через движок biber
%Сделана отдельная секция, чтобы не отображались в списке цитированных материалов
%    \begin{refsection}[vak,papers,conf]% Подсчет и нумерация авторских работ. Засчитываются только те, которые были прописаны внутри \nocite{}.
        %Чтобы сменить порядок разделов в сгрупированном списке литературы необходимо перетасовать следующие три строчки, а также команды в разделе \newcommand*{\insertbiblioauthorgrouped} в файле biblio/biblatex.tex
 %       \printbibliography[heading=countauthorvak, env=countauthorvak, keyword=biblioauthorvak, section=1]%
 %       \printbibliography[heading=countauthorconf, env=countauthorconf, keyword=biblioauthorconf, section=1]%
 %       \printbibliography[heading=countauthornotvak, env=countauthornotvak, keyword=biblioauthornotvak, section=1]%
 %       \printbibliography[heading=countauthor, env=countauthor, keyword=biblioauthor, section=1]%
%        \nocite{%Порядок перечисления в этом блоке определяет порядок вывода в списке публикаций автора
 %               vakbib1,vakbib2,%
 %               confbib1,confbib2,%
 %b              bib1,bib2,%
 %       }%
 %       \publications\ Основные результаты по теме диссертации изложены в~\arabic{citeauthor}~печатных изданиях, 
 %       \arabic{citeauthorvak} из которых изданы в журналах, рекомендованных ВАК, 
 %       \arabic{citeauthorconf} "--- в~тезисах докладов.
 %   \end{refsection}
%    \begin{refsection}[vak,papers,conf]%Блок, позволяющий отобрать из всех работ автора наиболее значимые, и только их вывести в автореферате, но считать в блоке выше общее число работ
%        \printbibliography[heading=countauthorvak, env=countauthorvak, keyword=biblioauthorvak, section=2]%
%        \printbibliography[heading=countauthornotvak, env=countauthornotvak, keyword=biblioauthornotvak, section=2]%
%        \printbibliography[heading=countauthorconf, env=countauthorconf, keyword=biblioauthorconf, section=2]%
%        \printbibliography[heading=countauthor, env=countauthor, keyword=biblioauthor, section=2]%
%        \nocite{vakbib2}%vak
%        \nocite{bib1}%notvak
%        \nocite{confbib1}%conf
%    \end{refsection}
%}
%При использовании пакета \verb!biblatex! для автоматического подсчёта
%количества публикаций автора по теме диссертации, необходимо
%их~здесь перечислить с использованием команды \verb!\nocite!.
