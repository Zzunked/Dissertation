
{\actuality} 
\paragraph{Актуальность темы исследования.}
В современной физике поверхности одной из основных задач является
создание низкоразмерных систем. После того, как графен был предсказан
теоретически~\cite{Wallace1947,McClure1956,Slonczewski1958}, а позже получен экспериментально~\cite{Novoselov2004}, интерес к 
низкоразмерным системам резко возрос среди ученых. Причиной этому является
серьезный потенциал для использовании в перспективных низкоразмерных 
устройствах, таких как квантовые компьютеры~\cite{Barends2014}, молекулярные переключатели~\cite{G.Joachim2000,Nitzan2003}. 
К наиболее интересным низкоразмерным нанообъектам относятся структуры на 
основе двумерных гексагональных кристаллов графена и нитрида бора.
Гексагональный нитрибора имеет огромный потенциал применения в различных
электрооптических приборах, например, ультрафиолетовые лазеры, 
свето-диоды, фотодетекторы~\cite{Zheng2008}, а так же в
конфигурациях вида graphen/h-BN/metal~\cite{Kamalakar2014,Kamalakar2016,Usachev_doc}.
Данная работа посвящена гексагональному нитриду бора(h-BN). 


Как и графен, гексагональный нитрид бора может быть получен путем каталитического разложения молекул прекурсора из газообразной 
фазы на активной подложке(метод CVD), в случае h-BN это
боразин~\cite{Kidambi2014,Paffett1990}. Нитрид бора выращенный на
металлической подложке методом CVD может деформироваться
переодическим образом, это показано в работах~\cite{Preobrajenski2008_Adsorption-inducedgapstatesofh-BNonmetalsurfaces,Brugger2009,Nagashima1995,Weng2016}. Рассогласование структуры нитрида
бора и подложки вместе с взаимодействием 2p$\pi$-состояний атомов монослоя
и n$d$-состояниями атомов переходного металла приводят к тому, что 
системы образуются из химически неэквивалентных участков нитрида бора и подложки.
Эти участки расположенные с пространственной периодичностью 
относительно друг друга, формируют двумерную сверхрешетку. Такие
сверхрешетки становятся интересными объектами для изучения возможности
управлять физическими свойствами монослоя нитрида бора с помощью
адсорбции и/или интеркаляции атомов или молекул~\cite{Ataca2010}.


{\aim} данной работы является исследование процесса окисления 
монослоя гексагонального нитрида бора при помощи \textbf{методов} 
рентгеновской фотоэлектронной спектроскопии(X-Ray Photoelectron 
Spectroscopy, XPS), дифракции медленных электронов(Low Energy 
Electron Diffraction, LEED) и рентгеновской спектроскопии
поглощения в области ближней тонкой структуры(Near-Edge 
X-Ray Absorption Fine Structure Spectroscopy, NEXAFS
spectroscopy). 

Для~достижения поставленной цели необходимо было решить следующие {\tasks}:
\begin{enumerate}
  \item Приготовление высококачественного монослоя h-BN на
  монокристаллической поверхности Co(0001) и его характеризация 
  методами спектроскопии и дифракции медленных электронов.
  \item Осуществление поэтапного окисления h-BN/Co(0001) в
  результате воздействия атомарного кислорода.
  \item Исследование состояния системы на каждом этапе окисления
  методами спектроскопии.
  \item Анализ полученных спектров поглощения и фотоэмиссии 
  остовных уровней и получение информации о механизме 
  взаимодействия атомарного кислорода и системы h-BN/Co(0001).
\end{enumerate}


%{\novelty}
%\begin{enumerate}
%  \item Впервые \ldots
%  \item Впервые \ldots
%  \item Было выполнено оригинальное исследование \ldots
%\end{enumerate}

%{\influence} \ldots

%{\methods} \ldots

%{\defpositions}
%\begin{enumerate}
%  \item Первое положение
%  \item Второе положение
%  \item Третье положение
%  \item Четвертое положение
%\end{enumerate}
%В папке Documents можно ознакомиться в решением совета из Томского ГУ
%в~файле \verb+Def_positions.pdf+, где обоснованно даются рекомендации
%по~формулировкам защищаемых положений. 

%{\reliability} полученных результатов обеспечивается \ldots \ Результаты находятся в соответствии с результатами, полученными другими авторами.


%{\probation}
%Основные результаты работы докладывались~на:
%перечисление основных конференций, симпозиумов и~т.\:п.

%{\contribution} Автор принимал активное участие \ldots

%\publications\ Основные результаты по теме диссертации изложены в ХХ печатных изданиях~\cite{Sokolov,Gaidaenko,Lermontov,Management},
%Х из которых изданы в журналах, рекомендованных ВАК~\cite{Sokolov,Gaidaenko}, 
%ХХ --- в тезисах докладов~\cite{Lermontov,Management}.

%\ifnumequal{\value{bibliosel}}{0}{% Встроенная реализация с загрузкой файла через движок bibtex8
%    \publications\ Основные результаты по теме диссертации изложены в XX печатных изданиях, 
%    X из которых изданы в журналах, рекомендованных ВАК, 
%    X "--- в тезисах докладов.%
%}{% Реализация пакетом biblatex через движок biber
%Сделана отдельная секция, чтобы не отображались в списке цитированных материалов
%    \begin{refsection}[vak,papers,conf]% Подсчет и нумерация авторских работ. Засчитываются только те, которые были прописаны внутри \nocite{}.
        %Чтобы сменить порядок разделов в сгрупированном списке литературы необходимо перетасовать следующие три строчки, а также команды в разделе \newcommand*{\insertbiblioauthorgrouped} в файле biblio/biblatex.tex
 %       \printbibliography[heading=countauthorvak, env=countauthorvak, keyword=biblioauthorvak, section=1]%
 %       \printbibliography[heading=countauthorconf, env=countauthorconf, keyword=biblioauthorconf, section=1]%
 %       \printbibliography[heading=countauthornotvak, env=countauthornotvak, keyword=biblioauthornotvak, section=1]%
 %       \printbibliography[heading=countauthor, env=countauthor, keyword=biblioauthor, section=1]%
%        \nocite{%Порядок перечисления в этом блоке определяет порядок вывода в списке публикаций автора
 %               vakbib1,vakbib2,%
 %               confbib1,confbib2,%
 %b              bib1,bib2,%
 %       }%
 %       \publications\ Основные результаты по теме диссертации изложены в~\arabic{citeauthor}~печатных изданиях, 
 %       \arabic{citeauthorvak} из которых изданы в журналах, рекомендованных ВАК, 
 %       \arabic{citeauthorconf} "--- в~тезисах докладов.
 %   \end{refsection}
%    \begin{refsection}[vak,papers,conf]%Блок, позволяющий отобрать из всех работ автора наиболее значимые, и только их вывести в автореферате, но считать в блоке выше общее число работ
%        \printbibliography[heading=countauthorvak, env=countauthorvak, keyword=biblioauthorvak, section=2]%
%        \printbibliography[heading=countauthornotvak, env=countauthornotvak, keyword=biblioauthornotvak, section=2]%
%        \printbibliography[heading=countauthorconf, env=countauthorconf, keyword=biblioauthorconf, section=2]%
%        \printbibliography[heading=countauthor, env=countauthor, keyword=biblioauthor, section=2]%
%        \nocite{vakbib2}%vak
%        \nocite{bib1}%notvak
%        \nocite{confbib1}%conf
%    \end{refsection}
%}
%При использовании пакета \verb!biblatex! для автоматического подсчёта
%количества публикаций автора по теме диссертации, необходимо
%их~здесь перечислить с использованием команды \verb!\nocite!.
