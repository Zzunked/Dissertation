%% Согласно ГОСТ Р 7.0.11-2011:
%% 5.3.3 В заключении диссертации излагают итоги выполненного исследования, рекомендации, перспективы дальнейшей разработки темы.
%% 9.2.3 В заключении автореферата диссертации излагают итоги данного исследования, рекомендации и перспективы дальнейшей разработки темы.
Основные результаты работы заключаются в следующем. Методами LEED, XPS и NEXAFS был исследован поэтапный процесс
окисления гексагонального нитрида бора на кобальте. 


На основе анализа спектров XPS и NEXAFS было показано:
\begin{enumerate}
  \item При окислении системы h-BN/Co(0001) атомы кислорода замещают атомы азота. Последние, после замещения,
  удаляются с поверхности монослоя h-BN. Атомы бора остаются в решетке монослоя.
  \item Встроенные в решетку атомы кислорода образуют связи с атомами бора, результатом этих связей
  является появление на поверхности структур $\mathrm{BN_2O}$ и $\mathrm{BO_3}$. Структуры $\mathrm{BNO_2}$ 
  практически не образовываются на поверхности.
  \item Встраивание атомов кислорода в решетку монослоя сопровождается интеркаляцией атомов кислорода
  под монослой нитрида бора, в результате чего, происходит экранирование монослоя h-BN от подложки Co(0001).
  \item  Процесс встраивания атомов кислорода в кристаллическую решетку гексагонального нитрида бора происходит
  более активно, чем процесс интеркаляции атомов кислорода под монослой h-BN.
\end{enumerate}
Анализ расчета модели окисленного монослоя h-BN показал следующее:
\begin{enumerate}
	\item Атомы кислорода замещают атомы азота в решетке монослоя случайным образом.
	\item Структура $\mathrm{BNO_2}$ оказывается нестабильной, в результате чего, не образуется
	на поверхности монослоя h-BN.
\end{enumerate}
