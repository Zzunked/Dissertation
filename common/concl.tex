%% Согласно ГОСТ Р 7.0.11-2011:
%% 5.3.3 В заключении диссертации излагают итоги выполненного исследования, рекомендации, перспективы дальнейшей разработки темы.
%% 9.2.3 В заключении автореферата диссертации излагают итоги данного исследования, рекомендации и перспективы дальнейшей разработки темы.
Основные результаты работы заключаются в следующем. Методами ДМЭ, РФЭС и NEXAFS был исследован поэтапный процесс
прогревания системы h-BN/Co(0001) в атмосфере молекулярного кислорода.


На основе анализа спектров РФЭС и NEXAFS было показано:

  \begin{enumerate}
    \vspace{15pt}
    \item При прогревании системы h-BN/Co(0001) в атмосфере кислорода, атомы кислорода замещают атомы азота. В результате происходит десорбция атомов азота с  поверхности монослоя h-BN. Атомы бора остаются в решетке монослоя.
    \item Встроенные в решетку атомы кислорода образуют связи с атомами бора, результатом этих связей
    является появление на поверхности структур $\mathrm{BN_2O}$ и $\mathrm{BO_3}$. Структуры $\mathrm{BNO_2}$ практически не образовываются на поверхности.
    \item Встраивание атомов кислорода в решетку монослоя сопровождается интеркаляцией атомов кислорода
    под монослой нитрида бора.
    \item Анализ статистической модели окисленного монослоя h-BN, а также DFT 
    расчет структуры показали, что структура $\mathrm{BNO_2}$ оказывается нестабильной, в 
    результате чего не образуется на поверхности монослоя h-BN.
  \end{enumerate}

  \vspace{15pt}

В заключении автор выражает благодарность научному руководителю Усачеву~Д.\:Ю. и аспирантам Бокай~К.\:А и
Шевелеву~В.\:О. за проведение измерений на Российско-Германском канале СИ синхротрона BESSY II в Берлине, а также обсуждение результатов.
