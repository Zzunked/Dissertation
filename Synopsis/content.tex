
\section*{Общая характеристика работы}

\newcommand{\actuality}{\underline{\textbf{\actualityTXT}}}
\newcommand{\progress}{\underline{\textbf{\progressTXT}}}
\newcommand{\aim}{\underline{{\textbf\aimTXT}}}
\newcommand{\tasks}{\underline{\textbf{\tasksTXT}}}
\newcommand{\novelty}{\underline{\textbf{\noveltyTXT}}}
\newcommand{\influence}{\underline{\textbf{\influenceTXT}}}
\newcommand{\methods}{\underline{\textbf{\methodsTXT}}}
\newcommand{\defpositions}{\underline{\textbf{\defpositionsTXT}}}
\newcommand{\reliability}{\underline{\textbf{\reliabilityTXT}}}
\newcommand{\probation}{\underline{\textbf{\probationTXT}}}
\newcommand{\contribution}{\underline{\textbf{\contributionTXT}}}
\newcommand{\publications}{\underline{\textbf{\publicationsTXT}}}


{\actuality} 
\paragraph{Актуальность темы исследования.}
В современной физике поверхности одной из основных задач является
создание низкоразмерных систем. После того, как графен был предсказан
теоретически~\cite{Wallace1947,McClure1956,Slonczewski1958}, а позже получен экспериментально~\cite{Novoselov2004}, интерес к 
низкоразмерным системам резко возрос среди ученых. Причиной этому является
серьезный потенциал для использовании в перспективных низкоразмерных 
устройствах, таких как квантовые компьютеры~\cite{Barends2014}, молекулярные переключатели~\cite{G.Joachim2000,Nitzan2003}. 
К наиболее интересным низкоразмерным нанообъектам относятся структуры на 
основе двумерных гексагональных кристаллов графена и нитрида бора.
Гексагональный нитрибора имеет огромный потенциал применения в различных
электрооптических приборах, например, ультрафиолетовые лазеры, 
свето-диоды, фотодетекторы~\cite{Zheng2008} и другие электронные
устройства, в основе которых лежат квантоворазмерные эффекты~\cite{
Chau2007}. Так же последнее время h-BN все больше привлекает внимание
ученых как изолятор в конфигурациях вида graphen/h-BN/metal~\cite{
Kamalakar2014,Kamalakar2016,Usachev_doc}.
Данная работа посвящена гексагональному нитриду бора(h-BN). 


Как и графен, гексагональный нитрид бора может быть получен путем каталитического разложения молекул прекурсора из газообразной 
фазы на активной подложке(метод CVD), в случае h-BN это
боразин~\cite{Kidambi2014,Paffett1990}. Нитрид бора выращенный на
металлической подложке методом CVD может деформироваться
переодическим образом, это показано в работах~\cite{Preobrajenski2008_Adsorption-inducedgapstatesofh-BNonmetalsurfaces,Brugger2009,Nagashima1995,Weng2016}. Рассогласование структуры нитрида
бора и подложки вместе с взаимодействием 2p$\pi$-состояний атомов монослоя
и n$d$-состояниями атомов переходного металла приводят к тому, что 
системы образуются из химически неэквивалентных участков нитрида бора и подложки.
Эти участки расположенные с пространственной периодичностью 
относительно друг друга, формируют двумерную сверхрешетку. Такие
сверхрешетки становятся интересными объектами для изучения возможности
управлять физическими свойствами монослоя нитрида бора с помощью
адсорбции и/или интеркаляции атомов или молекул~\cite{Ataca2010}.


{\aim} данной работы является исследование процесса окисления 
монослоя гексагонального нитрида бора при помощи \textbf{методов} 
рентгеновской фотоэлектронной спектроскопии(X-Ray Photoelectron 
Spectroscopy, XPS), дифракции медленных электронов(Low Energy 
Electron Diffraction, LEED) и рентгеновской спектроскопии
поглощения в области ближней тонкой структуры(Near-Edge 
X-Ray Absorption Fine Structure Spectroscopy, NEXAFS
spectroscopy). 

Для~достижения поставленной цели необходимо было решить следующие {\tasks}:
\begin{enumerate}
  \item Приготовление высококачественного монослоя h-BN на
  монокристаллической поверхности Co(0001) и его характеризация 
  методами спектроскопии и дифракции медленных электронов.
  \item Осуществление поэтапного окисления h-BN/Co(0001) в
  результате воздействия молекулярного кислорода.
  \item Исследование состояния системы на каждом этапе окисления
  методами спектроскопии.
  \item Анализ полученных спектров поглощения и фотоэмиссии 
  остовных уровней и получение информации о механизме 
  взаимодействия молекулярного кислорода и системы h-BN/Co(0001).
  \item Теоретический рассчет модели окисления гексагонального
  нитрида бора.
\end{enumerate}


%{\novelty}
%\begin{enumerate}
%  \item Впервые \ldots
%  \item Впервые \ldots
%  \item Было выполнено оригинальное исследование \ldots
%\end{enumerate}

%{\influence} \ldots

%{\methods} \ldots

%{\defpositions}
%\begin{enumerate}
%  \item Первое положение
%  \item Второе положение
%  \item Третье положение
%  \item Четвертое положение
%\end{enumerate}
%В папке Documents можно ознакомиться в решением совета из Томского ГУ
%в~файле \verb+Def_positions.pdf+, где обоснованно даются рекомендации
%по~формулировкам защищаемых положений. 

%{\reliability} полученных результатов обеспечивается \ldots \ Результаты находятся в соответствии с результатами, полученными другими авторами.


%{\probation}
%Основные результаты работы докладывались~на:
%перечисление основных конференций, симпозиумов и~т.\:п.

%{\contribution} Автор принимал активное участие \ldots

%\publications\ Основные результаты по теме диссертации изложены в ХХ печатных изданиях~\cite{Sokolov,Gaidaenko,Lermontov,Management},
%Х из которых изданы в журналах, рекомендованных ВАК~\cite{Sokolov,Gaidaenko}, 
%ХХ --- в тезисах докладов~\cite{Lermontov,Management}.

%\ifnumequal{\value{bibliosel}}{0}{% Встроенная реализация с загрузкой файла через движок bibtex8
%    \publications\ Основные результаты по теме диссертации изложены в XX печатных изданиях, 
%    X из которых изданы в журналах, рекомендованных ВАК, 
%    X "--- в тезисах докладов.%
%}{% Реализация пакетом biblatex через движок biber
%Сделана отдельная секция, чтобы не отображались в списке цитированных материалов
%    \begin{refsection}[vak,papers,conf]% Подсчет и нумерация авторских работ. Засчитываются только те, которые были прописаны внутри \nocite{}.
        %Чтобы сменить порядок разделов в сгрупированном списке литературы необходимо перетасовать следующие три строчки, а также команды в разделе \newcommand*{\insertbiblioauthorgrouped} в файле biblio/biblatex.tex
 %       \printbibliography[heading=countauthorvak, env=countauthorvak, keyword=biblioauthorvak, section=1]%
 %       \printbibliography[heading=countauthorconf, env=countauthorconf, keyword=biblioauthorconf, section=1]%
 %       \printbibliography[heading=countauthornotvak, env=countauthornotvak, keyword=biblioauthornotvak, section=1]%
 %       \printbibliography[heading=countauthor, env=countauthor, keyword=biblioauthor, section=1]%
%        \nocite{%Порядок перечисления в этом блоке определяет порядок вывода в списке публикаций автора
 %               vakbib1,vakbib2,%
 %               confbib1,confbib2,%
 %b              bib1,bib2,%
 %       }%
 %       \publications\ Основные результаты по теме диссертации изложены в~\arabic{citeauthor}~печатных изданиях, 
 %       \arabic{citeauthorvak} из которых изданы в журналах, рекомендованных ВАК, 
 %       \arabic{citeauthorconf} "--- в~тезисах докладов.
 %   \end{refsection}
%    \begin{refsection}[vak,papers,conf]%Блок, позволяющий отобрать из всех работ автора наиболее значимые, и только их вывести в автореферате, но считать в блоке выше общее число работ
%        \printbibliography[heading=countauthorvak, env=countauthorvak, keyword=biblioauthorvak, section=2]%
%        \printbibliography[heading=countauthornotvak, env=countauthornotvak, keyword=biblioauthornotvak, section=2]%
%        \printbibliography[heading=countauthorconf, env=countauthorconf, keyword=biblioauthorconf, section=2]%
%        \printbibliography[heading=countauthor, env=countauthor, keyword=biblioauthor, section=2]%
%        \nocite{vakbib2}%vak
%        \nocite{bib1}%notvak
%        \nocite{confbib1}%conf
%    \end{refsection}
%}
%При использовании пакета \verb!biblatex! для автоматического подсчёта
%количества публикаций автора по теме диссертации, необходимо
%их~здесь перечислить с использованием команды \verb!\nocite!.
 % Характеристика работы по структуре во введении и в автореферате не отличается (ГОСТ Р 7.0.11, пункты 5.3.1 и 9.2.1), потому её загружаем из одного и того же внешнего файла, предварительно задав форму выделения некоторым параметрам

%Диссертационная работа была выполнена при поддержке грантов ...

%\underline{\textbf{Объем и структура работы.}} Диссертация состоит из~введения, четырех глав, заключения и~приложения. Полный объем диссертации \textbf{ХХХ}~страниц текста с~\textbf{ХХ}~рисунками и~5~таблицами. Список литературы содержит \textbf{ХХX}~наименование.

%\newpage
\section*{Содержание работы}
Во \underline{\textbf{введении}} обосновывается актуальность
исследований, проводимых в~рамках данной диссертационной работы,
приводится обзор научной литературы по изучаемой проблеме,
формулируется цель, ставятся задачи работы, излагается научная новизна
и практическая значимость представляемой работы. В~последующих главах
сначала описывается общий принцип, позволяющий ..., а~потом идёт
апробация на частных примерах: ...  и~... .


\underline{\textbf{Первая глава}} посвящена ...

 картинку можно добавить так:
\begin{figure}[ht] 
  \centering
  \includegraphics [scale=0.27] {latex}
  \caption{Подпись к картинке.} 
  \label{img:latex}
\end{figure}

Формулы в строку без номера добавляются так:
\[ 
  \lambda_{T_s} = K_x\frac{d{x}}{d{T_s}}, \qquad
  \lambda_{q_s} = K_x\frac{d{x}}{d{q_s}},
\]

\underline{\textbf{Вторая глава}} посвящена исследованию 

\underline{\textbf{Третья глава}} посвящена исследованию

Можно сослаться на свои работы в автореферате. Для этого в файле
\verb!Synopsis/setup.tex! необходимо присвоить положительное значение
счётчику \verb!\setcounter{usefootcite}{1}!. В таком случае ссылки на
работы других авторов будут подстрочными.
\ifnumgreater{\value{usefootcite}}{0}{
Изложенные в третьей главе результаты опубликованы в~\cite{vakbib1, vakbib2}.
}{}
Использование подстрочных ссылок внутри таблиц может вызывать проблемы.

В \underline{\textbf{четвертой главе}} приведено описание 

В \underline{\textbf{заключении}} приведены основные результаты работы, которые заключаются в следующем:
%% Согласно ГОСТ Р 7.0.11-2011:
%% 5.3.3 В заключении диссертации излагают итоги выполненного исследования, рекомендации, перспективы дальнейшей разработки темы.
%% 9.2.3 В заключении автореферата диссертации излагают итоги данного исследования, рекомендации и перспективы дальнейшей разработки темы.
Основные результаты работы заключаются в следующем. Методами LEED, XPS и NEXAFS был исследован поэтапный процесс
окисления гексагонального нитрида бора на кобальте. 


На основе анализа спектров XPS и NEXAFS было показано:
\begin{enumerate}
  \item При окислении системы h-BN/Co(0001) атомы кислорода замещают атомы азота. Последние, после замещения,
  удаляются с поверхности монослоя h-BN. Атомы бора остаются в решетке монослоя.
  \item Встроенные в решетку атомы кислорода образуют связи с атомами бора, результатом этих связей
  является появление на поверхности структур $\mathrm{BN_2O}$ и $\mathrm{BO_3}$. Структуры $\mathrm{BNO_2}$ 
  практически не образовываются на поверхности.
  \item Встраивание атомов кислорода в решетку монослоя сопровождается интеркаляцией атомов кислорода
  под монослой нитрида бора, в результате чего, происходит экранирование монослоя h-BN от подложки Co(0001).
  \item  Процесс встраивания атомов кислорода в кристаллическую решетку гексагонального нитрида бора происходит
  более активно, чем процесс интеркаляции атомов кислорода под монослой h-BN.
\end{enumerate}
Анализ расчета модели окисленного монослоя h-BN показал следующее:
\begin{enumerate}
	\item Атомы кислорода замещают атомы азота в решетке монослоя случайным образом.
	\item Структура $\mathrm{BNO_2}$ оказывается нестабильной, в результате чего, не образуется
	на поверхности монослоя h-BN.
\end{enumerate}



%\newpage
При использовании пакета \verb!biblatex! список публикаций автора по теме
диссертации формируется в разделе <<\publications>>\ файла
\verb!../common/characteristic.tex!  при помощи команды \verb!\nocite! 

\ifdefmacro{\microtypesetup}{\microtypesetup{protrusion=false}}{} % не рекомендуется применять пакет микротипографики к автоматически генерируемому списку литературы
\ifnumequal{\value{bibliosel}}{0}{% Встроенная реализация с загрузкой файла через движок bibtex8
  \renewcommand{\bibname}{\large \authorbibtitle}
  \nocite{*}
  \insertbiblioauthor           % Подключаем Bib-базы
  %\insertbiblioother   % !!! bibtex не умеет работать с несколькими библиографиями !!!
}{% Реализация пакетом biblatex через движок biber
  \ifnumgreater{\value{usefootcite}}{0}{
%  \nocite{*} % Невидимая цитата всех работ, позволит вывести все работы автора
  \insertbiblioauthorcited      % Вывод процитированных в автореферате работ автора
  }{
  \insertbiblioauthor           % Вывод всех работ автора
%  \insertbiblioauthorgrouped    % Вывод всех работ автора, сгруппированных по источникам
%  \insertbiblioauthorimportant  % Вывод наиболее значимых работ автора (определяется в файле characteristic во второй section)
  \insertbiblioother            % Вывод списка литературы, на которую ссылались в тексте автореферата
  }
}
\ifdefmacro{\microtypesetup}{\microtypesetup{protrusion=true}}{}

