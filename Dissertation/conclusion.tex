\chapter*{Заключение}						% Заголовок
\addcontentsline{toc}{chapter}{Заключение}	% Добавляем его в оглавление

%% Согласно ГОСТ Р 7.0.11-2011:
%% 5.3.3 В заключении диссертации излагают итоги выполненного исследования, рекомендации, перспективы дальнейшей разработки темы.
%% 9.2.3 В заключении автореферата диссертации излагают итоги данного исследования, рекомендации и перспективы дальнейшей разработки темы.
%% Поэтому имеет смысл сделать эту часть общей и загрузить из одного файла в автореферат и в диссертацию:

% Основные результаты работы заключаются в следующем. Методами ФЭСУР была исследована электронная структура топологических состояний легированных металлами топологических изоляторов $Bi_{2}Se_{3}$, $Sb_{2}Te_{3}$ и $Bi_{x}Sb_{2-x}Te_{3}$ с разным типом легирующего металла (переходный магнитный \textit{Pb} и редкоземельный магнитный \textit{Gd}), с разной стехиометрией и положением уровня Ферми относительно точки Дирака.
% %%% Согласно ГОСТ Р 7.0.11-2011:
%% 5.3.3 В заключении диссертации излагают итоги выполненного исследования, рекомендации, перспективы дальнейшей разработки темы.
%% 9.2.3 В заключении автореферата диссертации излагают итоги данного исследования, рекомендации и перспективы дальнейшей разработки темы.
Основные результаты работы заключаются в следующем. Методами LEED, XPS и NEXAFS был исследован поэтапный процесс
окисления гексагонального нитрида бора на кобальте. 


На основе анализа спектров XPS и NEXAFS было показано:
\begin{enumerate}
  \item При окислении системы h-BN/Co(0001) атомы кислорода замещают атомы азота. Последние, после замещения,
  удаляются с поверхности монослоя h-BN. Атомы бора остаются в решетке монослоя.
  \item Встроенные в решетку атомы кислорода образуют связи с атомами бора, результатом этих связей
  является появление на поверхности структур $\mathrm{BN_2O}$ и $\mathrm{BO_3}$. Структуры $\mathrm{BNO_2}$ 
  практически не образовываются на поверхности.
  \item Встраивание атомов кислорода в решетку монослоя сопровождается интеркаляцией атомов кислорода
  под монослой нитрида бора, в результате чего, происходит экранирование монослоя h-BN от подложки Co(0001).
  \item  Процесс встраивания атомов кислорода в кристаллическую решетку гексагонального нитрида бора происходит
  более активно, чем процесс интеркаляции атомов кислорода под монослой h-BN.
\end{enumerate}
Анализ расчета модели окисленного монослоя h-BN показал следующее:
\begin{enumerate}
	\item Атомы кислорода замещают атомы азота в решетке монослоя случайным образом.
	\item Структура $\mathrm{BNO_2}$ оказывается нестабильной, в результате чего, не образуется
	на поверхности монослоя h-BN.
\end{enumerate}


% Результаты напыления ультратонких пленок свинца на поверхности топологических изоляторов $Bi_{2}Se_{3}$ и $Sb_{2}Te_{3}$ свидетельствуют о сохранении уникальной структуры дираковского конуса. Это говорит о возможности реализации контакта сверхпроводника и топологического изоляторая в данных системах. В ходе анализа полученных данных было  показано следующее:
% \begin{enumerate}
% 	\item Напыление немагнитного металла (\textit{Pb}) не нарушает уникальную структуру конуса Дирака, линейная дисперсия топологических состояний сохраняется;
% 	\item Напыление Pb приводит к возрастанию энергии связи дираковского конуса и внутренних уровней и одновременному перераспределению интенсивности электронных состояний системы в фотоэлектронных спектра;
% 	\item Посредством напыления слоя Pb возможно управление точкой Дирака в отсутствие всяких внешних полей. При этом обеспечивается устойчивое положение точки Дирака, в том числе и на уровне Ферми; 
% 	\item Анализируемые в работе данные свидетельствуют о появлении после напыления Pb электронных состоний, реализуемых посредством квантово-размерных эффектов в двумерном электронном газе.
% \end{enumerate}

% Анализ допированного \textit{Gd} топологического изолятора со стехиометрией $Bi_{1.09}Gd_{0.06}Sb_{0.85}Te_{3}$ показывает следующее:
% \begin{enumerate}
% 	\item Легирование небольшим количеством магнитной примеси (\textit{Gd}) существенно не нарушает линейную дисперсию ветвей конуса Дирака вдали от точки Дирака, однако в области точки Дирака дисперсия топологических состояний искажается вследствие магнитных свойств примеси;
% 	\item Для топологического изолятора, допированного магнитной примесью (\textit{Gd}), проанализирована вероятность наличия щели в электронной структуре Дираковского конуса в области точки Дирака. Ее величина в случае топологического изолятора $Bi_{1.09}Gd_{0.06}Sb_{0.85}Te_{3}$ составляет порядка 30 мэВ;
% 	\item Магнитные свойства допированного топологического изолятора напрямую связаны со свойствами электронной структуры образца;
% 	\item Наличие антиферромагнитного порядка в объеме Gd-допированного топологического изолятора, существующего при низких температурах и нарушающегося по мере роста температуры;
% 	\item Поверхностный слой Gd-допированного топологического изолятора при высоких температурах (порядка 100 К) демонстрирует ферромагнитные свойства, обусловленные отличными от объема особенностями электронной структуры поверхности образца, которые могут быть описаны в рамках модели RKKY взаимодействия.
% \end{enumerate}

%В заключение автор выражает благодарность и большую признательность научному руководителю Шикину~А.\:М. и старшему научному сотруднику Климовских~И.\:И., за поддержку, помощь, обсуждение результатов и научное руководство. Также автор благодарит Королеву~А.\:В. за помощь в~работе с~образцами, обработке и~обсуждении результатов.